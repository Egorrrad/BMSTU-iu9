% Поля
%--------------------------------------
\usepackage{geometry}
\geometry{a4paper,tmargin=2cm,bmargin=2cm,lmargin=3cm,rmargin=1cm}
%--------------------------------------


%Russian-specific packages
%--------------------------------------
\usepackage[T2A]{fontenc}
\usepackage[utf8]{inputenc}
\usepackage[english, main=russian]{babel}

%--------------------------------------

\usepackage{textcomp}

% Красная строка
%--------------------------------------
\usepackage{indentfirst}               
%--------------------------------------             


%Graphics
%--------------------------------------
\usepackage{graphicx}
\graphicspath{ {./images/} }
\usepackage{float}%"Плавающие" картинки
%--------------------------------------

% Полуторный интервал
%--------------------------------------
\linespread{1.3}                    
%--------------------------------------

%Выравнивание и переносы
%--------------------------------------
% Избавляемся от переполнений
\sloppy
% Запрещаем разрыв страницы после первой строки абзаца
\clubpenalty=10000
% Запрещаем разрыв страницы после последней строки абзаца
\widowpenalty=10000
%--------------------------------------

%Списки
\usepackage{enumitem}

%Подписи
\usepackage{caption} 

%Гиперссылки
% \usepackage{hyperref}

\usepackage[colorlinks, urlcolor = blue, filecolor = blue, citecolor = blue, linkcolor = black]{hyperref}


\hypersetup {
unicode=true
}

%Рисунки
%--------------------------------------
\DeclareCaptionLabelSeparator*{emdash}{~--- }
\captionsetup[figure]{labelsep=emdash,font=onehalfspacing,position=bottom}
%--------------------------------------

\usepackage{tempora}

%Листинги
%--------------------------------------
\usepackage{listings} % Пакет для отображения кода
\usepackage{xcolor} % Пакет для цветных элементов

\lstset{inputencoding=utf8, extendedchars=false, keepspaces = true}

\definecolor{codegreen}{rgb}{0,0.6,0}
\definecolor{codegray}{rgb}{0.5,0.5,0.5}
\definecolor{codepurple}{rgb}{0.68,0.8,0.82}
\definecolor{backcolour}{rgb}{0.9,0.9,0.9} 
\definecolor{keyword}{rgb}{0.93, 0.68, 0.18}   % Оранжевые ключевые слова

\lstdefinestyle{mystyle}{
    backgroundcolor=\color{backcolour},   % цвет фона
    commentstyle=\color{codegreen},       % цвет комментариев
    keywordstyle=\color{keyword},        % цвет ключевых слов
    numberstyle=\tiny\color{codegray},    % стиль нумерации строк
    stringstyle=\color{codegreen},       % цвет строк
    basicstyle=\ttfamily\footnotesize,    % основной стиль текста
    breakatwhitespace=false,              % перенос по пробелам
    breaklines=true,                      % автоматический перенос строк
    captionpos=b,                         % позиция заголовка внизу
    keepspaces=true,                      % сохранять пробелы
    numbers=left,                         % нумерация строк слева
    numbersep=5pt,                        % отступ нумерации
    showspaces=false,                     % скрывать пробелы
    showstringspaces=false,               % скрывать пробелы в строках
    showtabs=false,                       % скрывать табуляцию
    tabsize=2                             % размер табуляции
}

\lstset{style=mystyle, inputencoding=utf8}



%--------------------------------------

%%% Математические пакеты %%%
%--------------------------------------
\usepackage{amsthm,amsfonts,amsmath,amssymb,amscd}  % Математические дополнения от AMS
\usepackage{mathtools}                              % Добавляет окружение multlined
\usepackage[perpage]{footmisc}


% графики
\usepackage{booktabs}        % Пакет для улучшенных таблиц
\usepackage{pgfplots}        % Пакет для построения графиков
\pgfplotsset{compat=1.17}    % Установка совместимости

% Литература 

% переименовываем  список литературы в "список используемой литературы"
\addto\captionsrussian{\def\refname{Список литературы}}



