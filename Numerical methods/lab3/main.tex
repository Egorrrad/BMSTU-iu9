\documentclass[a4paper, 14pt]{extarticle}

% Поля
%--------------------------------------
\usepackage{geometry}
\geometry{a4paper,tmargin=2cm,bmargin=2cm,lmargin=3cm,rmargin=1cm}
%--------------------------------------


%Russian-specific packages
%--------------------------------------
\usepackage[T2A]{fontenc}
\usepackage[utf8]{inputenc}
\usepackage[english, main=russian]{babel}

%--------------------------------------

\usepackage{textcomp}

% Красная строка
%--------------------------------------
\usepackage{indentfirst}               
%--------------------------------------             


%Graphics
%--------------------------------------
\usepackage{graphicx}
\graphicspath{ {./images/} }
\usepackage{float}%"Плавающие" картинки
%--------------------------------------

% Полуторный интервал
%--------------------------------------
\linespread{1.3}                    
%--------------------------------------

%Выравнивание и переносы
%--------------------------------------
% Избавляемся от переполнений
\sloppy
% Запрещаем разрыв страницы после первой строки абзаца
\clubpenalty=10000
% Запрещаем разрыв страницы после последней строки абзаца
\widowpenalty=10000
%--------------------------------------

%Списки
\usepackage{enumitem}

%Подписи
\usepackage{caption} 

%Гиперссылки
% \usepackage{hyperref}

\usepackage[colorlinks, urlcolor = blue, filecolor = blue, citecolor = blue, linkcolor = black]{hyperref}


\hypersetup {
unicode=true
}

%Рисунки
%--------------------------------------
\DeclareCaptionLabelSeparator*{emdash}{~--- }
\captionsetup[figure]{labelsep=emdash,font=onehalfspacing,position=bottom}
%--------------------------------------

\usepackage{tempora}

%Листинги
%--------------------------------------
\usepackage{listings} % Пакет для отображения кода
\usepackage{xcolor} % Пакет для цветных элементов

\lstset{inputencoding=utf8, extendedchars=false, keepspaces = true}

\definecolor{codegreen}{rgb}{0,0.6,0}
\definecolor{codegray}{rgb}{0.5,0.5,0.5}
\definecolor{codepurple}{rgb}{0.68,0.8,0.82}
\definecolor{backcolour}{rgb}{0.9,0.9,0.9} 
\definecolor{keyword}{rgb}{0.93, 0.68, 0.18}   % Оранжевые ключевые слова

\lstdefinestyle{mystyle}{
    backgroundcolor=\color{backcolour},   % цвет фона
    commentstyle=\color{codegreen},       % цвет комментариев
    keywordstyle=\color{keyword},        % цвет ключевых слов
    numberstyle=\tiny\color{codegray},    % стиль нумерации строк
    stringstyle=\color{codegreen},       % цвет строк
    basicstyle=\ttfamily\footnotesize,    % основной стиль текста
    breakatwhitespace=false,              % перенос по пробелам
    breaklines=true,                      % автоматический перенос строк
    captionpos=b,                         % позиция заголовка внизу
    keepspaces=true,                      % сохранять пробелы
    numbers=left,                         % нумерация строк слева
    numbersep=5pt,                        % отступ нумерации
    showspaces=false,                     % скрывать пробелы
    showstringspaces=false,               % скрывать пробелы в строках
    showtabs=false,                       % скрывать табуляцию
    tabsize=2                             % размер табуляции
}

\lstset{style=mystyle, inputencoding=utf8}



%--------------------------------------

%%% Математические пакеты %%%
%--------------------------------------
\usepackage{amsthm,amsfonts,amsmath,amssymb,amscd}  % Математические дополнения от AMS
\usepackage{mathtools}                              % Добавляет окружение multlined
\usepackage[perpage]{footmisc}


% графики
\usepackage{booktabs}        % Пакет для улучшенных таблиц
\usepackage{pgfplots}        % Пакет для построения графиков
\pgfplotsset{compat=1.17}    % Установка совместимости
\usepackage{multirow}        % Для объединения строк
\usepackage{graphicx} % Для поворота текста
\usepackage{array}    % Для настройки колонок
\usepackage{rotating} 

% Литература 

% переименовываем  список литературы в "список используемой литературы"
\addto\captionsrussian{\def\refname{Список литературы}}






%--------------------------------------

%--------------------------------------
%			НАЧАЛО ДОКУМЕНТА
%--------------------------------------

\begin{document}

%--------------------------------------
%			ТИТУЛЬНЫЙ ЛИСТ
%--------------------------------------
\begin{titlepage}
\thispagestyle{empty}
\newpage


%Шапка титульного листа
%--------------------------------------
\vspace*{-60pt}
\hspace{-65pt}
\begin{minipage}{0.3\textwidth}
\hspace*{-20pt}\centering
\includegraphics[width=\textwidth]{emblem}
\end{minipage}
\begin{minipage}{0.67\textwidth}\small \textbf{
\vspace*{-0.7ex}
\hspace*{-6pt}\centerline{Министерство науки и высшего образования Российской Федерации}
\vspace*{-0.7ex}
\centerline{Федеральное государственное автономное образовательное учреждение }
\vspace*{-0.7ex}
\centerline{высшего образования}
\vspace*{-0.7ex}
\centerline{<<Московский государственный технический университет}
\vspace*{-0.7ex}
\centerline{имени Н.Э. Баумана}
\vspace*{-0.7ex}
\centerline{(национальный исследовательский университет)>>}
\vspace*{-0.7ex}
\centerline{(МГТУ им. Н.Э. Баумана)}}
\end{minipage}
%--------------------------------------

%Полосы
%--------------------------------------
\vspace{-25pt}
\hspace{-35pt}\rule{\textwidth}{2.3pt}

\vspace*{-20.3pt}
\hspace{-35pt}\rule{\textwidth}{0.4pt}
%--------------------------------------

\vspace{1.5ex}
\hspace{-35pt} \noindent \small ФАКУЛЬТЕТ\hspace{80pt} <<Информатика и системы управления>>

\vspace*{-16pt}
\hspace{47pt}\rule{0.83\textwidth}{0.4pt}

\vspace{0.5ex}
\hspace{-35pt} \noindent \small КАФЕДРА\hspace{50pt} <<Теоретическая информатика и компьютерные технологии>>

\vspace*{-16pt}
\hspace{30pt}\rule{0.866\textwidth}{0.4pt}

\vspace{11em}

\begin{center}
\Large {\bf Лабораторная работа №7} \\ 
\large {\bf по курсу <<Численные методы>>} \\
\large <<Тригонометрическая интерполиция функций
с помощью быстрого преобразования Фурье>> \\

\end{center}\normalsize

\vspace{8em}

\vbox{%
\hfill%
\vbox{%
\hbox{Студент:  }%
\hbox{Группа: ИУ9-61Б}%
\hbox{Преподаватель: Домрачева А.Б.}%
}%
} 

\bigskip

\vfill


\begin{center}
\textsl{Москва 2025}
\end{center}
\end{titlepage}
%--------------------------------------
%		КОНЕЦ ТИТУЛЬНОГО ЛИСТА
%--------------------------------------


\renewcommand{\ttdefault}{pcr}

\setlength{\tabcolsep}{3pt}


% Содержание:
%\tableofcontents
%\newpage

\newpage
\setcounter{page}{2}



\section{Постановка задачи}

\textbf{Дано:} Задача Коши для дифференциального уравнения второго порядка (вариант 4):

\[
y'' - y' = 2(1 - x), \quad y(0) = 1, \quad y'(0) = 1
\]

на отрезке $[0; 1]$.

\textbf{Найти:} 
\begin{enumerate}
\item Численное решение с погрешностью $\varepsilon = 0.001$
\item Точное аналитическое решение
\item Сравнение приближенного и точного решений на каждом шаге
\end{enumerate}

\section{Основные теоретические сведения}

\subsection{Методы Рунге-Кутта для решения СОДУ}
Методы Рунге-Кутта применяются для решения задачи Коши для системы обыкновенных дифференциальных уравнений первого порядка:

\[
\begin{cases}
y_1' = f_1(x, y_1, y_2, \ldots, y_n) \\
y_2' = f_2(x, y_1, y_2, \ldots, y_n) \\
\vdots \\
y_n' = f_n(x, y_1, y_2, \ldots, y_n)
\end{cases}
\]

на отрезке $[x_0, x_{end}]$ с начальными условиями:
\[
y_1(x_0) = y_{01}, \ldots, y_n(x_0) = y_{0n}
\]

В векторной форме:
\[
\mathbf{y}' = \mathbf{f}(x, \mathbf{y}), \quad \mathbf{y}(x_0) = \mathbf{y}_0
\]
где:
\begin{itemize}
\item $\mathbf{y} = (y_1(x), \ldots, y_n(x))$ - вектор искомых функций
\item $\mathbf{f} = (f_1(x, \mathbf{y}), \ldots, f_n(x, \mathbf{y}))$ - вектор правых частей
\item $\mathbf{y}_0 = (y_{01}, \ldots, y_{0n})$ - вектор начальных условий
\end{itemize}


\subsection{Классический метод Рунге-Кутты 4-го порядка}
Метод заключается в последовательном вычислении коэффициентов:

\[
\begin{cases}
K_1 = \mathbf{f}(x, \mathbf{y}) \\
K_2 = \mathbf{f}\left(x + \frac{h}{2}, \mathbf{y} + \frac{h}{2}K_1\right) \\
K_3 = \mathbf{f}\left(x + \frac{h}{2}, \mathbf{y} + \frac{h}{2}K_2\right) \\
K_4 = \mathbf{f}(x + h, \mathbf{y} + hK_3)
\end{cases}
\]

Приближенное решение в точке $x + h$:
\[
\mathbf{y}(x + h) \approx \mathbf{y}_h = \mathbf{y} + \frac{h}{6}(K_1 + 2K_2 + 2K_3 + K_4)
\]

Метод имеет четвертый порядок точности:
\[
\|\mathbf{y}(x + h) - \mathbf{y}_h\| \leq Ch^5
\]
где $C$ - константа, зависящая от правых частей системы.

\subsection{Автоматический выбор шага}
Для контроля точности используется следующий алгоритм:

\begin{enumerate}
\item Вычисляют два приближения:
\begin{itemize}
\item $y_h$ - два шага длины $h$
\item $y_{2h}$ - один шаг длины $2h$
\end{itemize}

\item Оценивают погрешность по правилу Рунге:
\[
err = \frac{1}{2^p - 1}\|y_h - y_{2h}\|
\]
где $p = 4$ - порядок точности метода

\item Вычисляют оптимальный шаг:
\[
h_{opt} = h\left(\frac{\epsilon}{err}\right)^{\frac{1}{p+1}}
\]

\item Если $err \leq \epsilon$, то:
\begin{itemize}
\item Принимают решение $y_h$ или уточненное значение $y_h + \frac{y_h - y_{2h}}{2^p - 1}$
\item Новый шаг $h_{new} = 0.9h_{opt}$
\end{itemize}

\item Если $err > \epsilon$, шаг уменьшают и повторяют вычисления
\end{enumerate}


\subsection{Приведение к системе ОДУ первого порядка}
Введем замену переменных:
\[
\begin{cases}
y_1 = y \\
y_2 = y'
\end{cases}
\]

Получаем систему:
\[
\begin{cases}
y_1' = y_2 \\
y_2' = 2(1 - x) + y_2
\end{cases}
\]
с начальными условиями $y_1(0) = 1$, $y_2(0) = 1$.

\section{Реализация}

\lstinputlisting[language=go, caption=Метод Рунге-Кутта с адаптивным шагом]{code/lab3.go}
\lstinputlisting[language=go, caption=Метод Рунге-Кутта с постоянным шагом]{code/lab3_constH.go}

\section{Результаты}

\begin{table}[h]
    \centering
    \caption{Автоматический подбор шага метода Рунге-Кутта}
    \label{tab:rk4}
    \begin{tabular}{cccccc}
    \toprule
    $x$ & Приближенное $y(x)$ & Точное $y(x)$ & Ошибка & Погрешность & Шаг \\
    \midrule
    0.00000 & 1.0000000 & 1.0000000 & 0.0000000 & 0.0000000 & 1.00000 \\
    0.13978 & 1.1695586 & 1.1695591 & 0.0000005 & 0.0008018 & 0.13978 \\
    0.27956 & 1.4007001 & 1.4007012 & 0.0000010 & 0.0008018 & 0.13978 \\
    0.41104 & 1.6773458 & 1.6773475 & 0.0000016 & 0.0006637 & 0.13148 \\
    0.54253 & 2.0146819 & 2.0146842 & 0.0000024 & 0.0006637 & 0.13148 \\
    0.67097 & 2.4063431 & 2.4063463 & 0.0000032 & 0.0006177 & 0.12845 \\
    0.79942 & 2.8633189 & 2.8633232 & 0.0000042 & 0.0006177 & 0.12845 \\
    0.92672 & 3.3849988 & 3.3850042 & 0.0000054 & 0.0006010 & 0.12730 \\
    \bottomrule
    \end{tabular}
    \end{table}


    \begin{table}[h]
        \centering
        \caption{Метод Рунге-Кутта с постоянным шагом}
        \label{tab:rk4_full}
        \small % Увеличиваем размер шрифта
        \begin{tabular}{
            >{\centering\arraybackslash}m{1.2cm} % Колонка "Проход"
            c
            c
            c
            c
            c
            c
        }
        \toprule
        \rotatebox{90}{\scriptsize Проход} & $x$ & Приближенное $y(x)$ & Точное $y(x)$ & Ошибка & Погрешность & Шаг \\
        \midrule
        \multirow{2}{*}{\rotatebox{90}{\scriptsize 0}} 
        & 0.00000 & 1.0000000 & 1.0000000 & 0.0000000 & 0.0000000 & 1.00000 \\
        & 1.00000 & 3.7083333 & 3.7182818 & 0.0099485 & 0.5556713 & 1.00000 \\
        \cmidrule(r){2-7}
        \multirow{3}{*}{\rotatebox{90}{\scriptsize 1}} 
        & 0.00000 & 1.0000000 & 1.0000000 & 0.0000000 & 0.0000000 & 0.50000 \\
        & 0.50000 & 1.8984375 & 1.8987213 & 0.0002838 & 0.0478231 & 0.50000 \\
        & 1.00000 & 3.7173462 & 3.7182818 & 0.0009356 & 0.0478231 & 0.50000 \\
        \cmidrule(r){2-7}
        \multirow{5}{*}{\rotatebox{90}{\scriptsize 2}} 
        & 0.00000 & 1.0000000 & 1.0000000 & 0.0000000 & 0.0000000 & 0.25000 \\
        & 0.25000 & 1.3465169 & 1.3465254 & 0.0000085 & 0.0049654 & 0.25000 \\
        & 0.50000 & 1.8986995 & 1.8987213 & 0.0000218 & 0.0049654 & 0.25000 \\
        & 0.75000 & 2.6794580 & 2.6795000 & 0.0000420 & 0.0049767 & 0.25000 \\
        & 1.00000 & 3.7182099 & 3.7182818 & 0.0000719 & 0.0049767 & 0.25000 \\
        \cmidrule(r){2-7}
        \multirow{9}{*}{\rotatebox{90}{\scriptsize 3}} 
        & 0.00000 & 1.0000000 & 1.0000000 & 0.0000000 & 0.0000000 & 0.12500 \\
        & 0.12500 & 1.1487732 & 1.1487735 & 0.0000003 & 0.0005675 & 0.12500 \\
        & 0.25000 & 1.3465248 & 1.3465254 & 0.0000006 & 0.0005675 & 0.12500 \\
        & 0.37500 & 1.5956154 & 1.5956164 & 0.0000010 & 0.0005676 & 0.12500 \\
        & 0.50000 & 1.8987198 & 1.8987213 & 0.0000015 & 0.0005676 & 0.12500 \\
        & 0.62500 & 2.2588688 & 2.2588710 & 0.0000021 & 0.0005678 & 0.12500 \\
        & 0.75000 & 2.6794971 & 2.6795000 & 0.0000029 & 0.0005678 & 0.12500 \\
        & 0.87500 & 3.1644964 & 3.1645003 & 0.0000038 & 0.0005681 & 0.12500 \\
        & 1.00000 & 3.7182768 & 3.7182818 & 0.0000050 & 0.0005681 & 0.12500 \\
        \bottomrule
        \end{tabular}
        \end{table}
        

\section{Вывод}
В ходе выполнения лабораторной работы была решена задача Коши для дифференциального уравнения второго порядка методом 
Рунге–Кутты четвёртого порядка. Уравнение было приведено к системе первого порядка, после чего были реализованы 
два подхода: с постоянным шагом и с автоматической адаптацией шага в зависимости от заданной точности. 
Полученные численные решения сравнивались с точным аналитическим решением, что позволило наглядно оценить 
точность каждого метода. Использование адаптивного шага позволило достичь требуемой точности $\varepsilon = 0.001$ 
при меньшем числе итераций по сравнению с методом с постоянным шагом, обеспечив более эффективное использование 
вычислительных ресурсов.

\end{document}