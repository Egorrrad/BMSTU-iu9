\documentclass[a4paper, 14pt]{extarticle}

% Поля
%--------------------------------------
\usepackage{geometry}
\geometry{a4paper,tmargin=2cm,bmargin=2cm,lmargin=3cm,rmargin=1cm}
%--------------------------------------


%Russian-specific packages
%--------------------------------------
\usepackage[T2A]{fontenc}
\usepackage[utf8]{inputenc}
\usepackage[english, main=russian]{babel}

%--------------------------------------

\usepackage{textcomp}

% Красная строка
%--------------------------------------
\usepackage{indentfirst}               
%--------------------------------------             


%Graphics
%--------------------------------------
\usepackage{graphicx}
\graphicspath{ {./images/} }
\usepackage{float}%"Плавающие" картинки
%--------------------------------------

% Полуторный интервал
%--------------------------------------
\linespread{1.3}                    
%--------------------------------------

%Выравнивание и переносы
%--------------------------------------
% Избавляемся от переполнений
\sloppy
% Запрещаем разрыв страницы после первой строки абзаца
\clubpenalty=10000
% Запрещаем разрыв страницы после последней строки абзаца
\widowpenalty=10000
%--------------------------------------

%Списки
\usepackage{enumitem}

%Подписи
\usepackage{caption} 

%Гиперссылки
% \usepackage{hyperref}

\usepackage[colorlinks, urlcolor = blue, filecolor = blue, citecolor = blue, linkcolor = black]{hyperref}


\hypersetup {
unicode=true
}

%Рисунки
%--------------------------------------
\DeclareCaptionLabelSeparator*{emdash}{~--- }
\captionsetup[figure]{labelsep=emdash,font=onehalfspacing,position=bottom}
%--------------------------------------

\usepackage{tempora}

%Листинги
%--------------------------------------
\usepackage{listings} % Пакет для отображения кода
\usepackage{xcolor} % Пакет для цветных элементов

\lstset{inputencoding=utf8, extendedchars=false, keepspaces = true}

\definecolor{codegreen}{rgb}{0,0.6,0}
\definecolor{codegray}{rgb}{0.5,0.5,0.5}
\definecolor{codepurple}{rgb}{0.68,0.8,0.82}
\definecolor{backcolour}{rgb}{0.9,0.9,0.9} 
\definecolor{keyword}{rgb}{0.93, 0.68, 0.18}   % Оранжевые ключевые слова

\lstdefinestyle{mystyle}{
    backgroundcolor=\color{backcolour},   % цвет фона
    commentstyle=\color{codegreen},       % цвет комментариев
    keywordstyle=\color{keyword},        % цвет ключевых слов
    numberstyle=\tiny\color{codegray},    % стиль нумерации строк
    stringstyle=\color{codegreen},       % цвет строк
    basicstyle=\ttfamily\footnotesize,    % основной стиль текста
    breakatwhitespace=false,              % перенос по пробелам
    breaklines=true,                      % автоматический перенос строк
    captionpos=b,                         % позиция заголовка внизу
    keepspaces=true,                      % сохранять пробелы
    numbers=left,                         % нумерация строк слева
    numbersep=5pt,                        % отступ нумерации
    showspaces=false,                     % скрывать пробелы
    showstringspaces=false,               % скрывать пробелы в строках
    showtabs=false,                       % скрывать табуляцию
    tabsize=2                             % размер табуляции
}

\lstset{style=mystyle, inputencoding=utf8}



%--------------------------------------

%%% Математические пакеты %%%
%--------------------------------------
\usepackage{amsthm,amsfonts,amsmath,amssymb,amscd}  % Математические дополнения от AMS
\usepackage{mathtools}                              % Добавляет окружение multlined
\usepackage[perpage]{footmisc}


% графики
\usepackage{booktabs}        % Пакет для улучшенных таблиц
\usepackage{pgfplots}        % Пакет для построения графиков
\pgfplotsset{compat=1.17}    % Установка совместимости
\usepackage{multirow}        % Для объединения строк
\usepackage{graphicx} % Для поворота текста
\usepackage{array}    % Для настройки колонок
\usepackage{rotating} 

% Литература 

% переименовываем  список литературы в "список используемой литературы"
\addto\captionsrussian{\def\refname{Список литературы}}






%--------------------------------------

%--------------------------------------
%			НАЧАЛО ДОКУМЕНТА
%--------------------------------------

\begin{document}

%--------------------------------------
%			ТИТУЛЬНЫЙ ЛИСТ
%--------------------------------------
\begin{titlepage}
\thispagestyle{empty}
\newpage


%Шапка титульного листа
%--------------------------------------
\vspace*{-60pt}
\hspace{-65pt}
\begin{minipage}{0.3\textwidth}
\hspace*{-20pt}\centering
\includegraphics[width=\textwidth]{emblem}
\end{minipage}
\begin{minipage}{0.67\textwidth}\small \textbf{
\vspace*{-0.7ex}
\hspace*{-6pt}\centerline{Министерство науки и высшего образования Российской Федерации}
\vspace*{-0.7ex}
\centerline{Федеральное государственное автономное образовательное учреждение }
\vspace*{-0.7ex}
\centerline{высшего образования}
\vspace*{-0.7ex}
\centerline{<<Московский государственный технический университет}
\vspace*{-0.7ex}
\centerline{имени Н.Э. Баумана}
\vspace*{-0.7ex}
\centerline{(национальный исследовательский университет)>>}
\vspace*{-0.7ex}
\centerline{(МГТУ им. Н.Э. Баумана)}}
\end{minipage}
%--------------------------------------

%Полосы
%--------------------------------------
\vspace{-25pt}
\hspace{-35pt}\rule{\textwidth}{2.3pt}

\vspace*{-20.3pt}
\hspace{-35pt}\rule{\textwidth}{0.4pt}
%--------------------------------------

\vspace{1.5ex}
\hspace{-35pt} \noindent \small ФАКУЛЬТЕТ\hspace{80pt} <<Информатика и системы управления>>

\vspace*{-16pt}
\hspace{47pt}\rule{0.83\textwidth}{0.4pt}

\vspace{0.5ex}
\hspace{-35pt} \noindent \small КАФЕДРА\hspace{50pt} <<Теоретическая информатика и компьютерные технологии>>

\vspace*{-16pt}
\hspace{30pt}\rule{0.866\textwidth}{0.4pt}

\vspace{11em}

\begin{center}
\Large {\bf Лабораторная работа №7} \\ 
\large {\bf по курсу <<Численные методы>>} \\
\large <<Тригонометрическая интерполиция функций
с помощью быстрого преобразования Фурье>> \\

\end{center}\normalsize

\vspace{8em}

\vbox{%
\hfill%
\vbox{%
\hbox{Студент:  }%
\hbox{Группа: ИУ9-61Б}%
\hbox{Преподаватель: Домрачева А.Б.}%
}%
} 

\bigskip

\vfill


\begin{center}
\textsl{Москва 2025}
\end{center}
\end{titlepage}
%--------------------------------------
%		КОНЕЦ ТИТУЛЬНОГО ЛИСТА
%--------------------------------------


\renewcommand{\ttdefault}{pcr}

\setlength{\tabcolsep}{3pt}


% Содержание:
%\tableofcontents
%\newpage

\newpage
\setcounter{page}{2}


\section{Постановка задачи}

\textbf{Дано:} Функция двух переменных:
\[ f(x_1, x_2) = \exp(x_1) + (x_1 + x_2)^2. \]

Начальное приближение: \( x^0 = (1, 1) \).

\textbf{Найти:}
\begin{enumerate}
    \item Найти минимум функции с точностью \( \varepsilon = 0.001 \) методом наискорейшего спуска.
    \item Найти минимум аналитически.
    \item Сравнить полученные результаты с аналитическим решением.
\end{enumerate}

\section{Основные теоретические сведения}

\subsection{Метод наискорейшего спуска}

Метод наискорейшего спуска — это итерационный метод, используемый для нахождения минимума функции. 
Для функции \( f(x_1, x_2) \) процесс выглядит следующим образом:

\begin{enumerate}
    \item На \( k \)-м шаге вычисляется градиент функции в точке \( x^k \):
    \[
    \nabla f(x^k) = \left( \frac{\partial f}{\partial x_1}(x^k), \frac{\partial f}{\partial x_2}(x^k) \right).
    \]
    \item Проверяется условие остановки:
    \[
    \|\nabla f(x^k)\| = \max_{1 \leq i \leq n} \left| \frac{\partial f}{\partial x_i}(x^k) \right| < \varepsilon.
    \]
    \item Если условие остановки не выполнено, определяется направление спуска \( d^k = -\nabla f(x^k) \).
    \item Рассматривается функция одной переменной:
    \[
    \varphi_k(t) = f(x^k + t \cdot d^k) = f(x^k - t \nabla f(x^k)).
    \]
    \item Находится \( t^* \), минимизирующее \( \varphi_k(t) \), с помощью метода одномерной оптимизации.
    \item Обновляется точка:
    \[
    x^{k+1} = x^k + t^* \cdot d^k.
    \]
\end{enumerate}

В данном случае для поиска \( t^* \) используется метод парабол, который аппроксимирует \( \varphi_k(t) \) квадратичной функцией и 
находит её минимум.

\section{Аналитическое решение}

Найдём минимум функции \( f(x_1, x_2) = \exp(x_1) + (x_1 + x_2)^2 \) аналитически.

Градиент:

\[
\frac{\partial f}{\partial x_1} = \exp(x_1) + 2(x_1 + x_2), \quad \frac{\partial f}{\partial x_2} = 2(x_1 + x_2).
\]

Приравниваем градиент к нулю для поиска стационарных точек:

\[
\begin{cases}
\exp(x_1) + 2(x_1 + x_2) = 0, \\
2(x_1 + x_2) = 0.
\end{cases}
\]

Из второго уравнения: \( x_1 + x_2 = 0 \implies x_2 = -x_1 \).

Подставляем \( x_2 = -x_1 \) в первое уравнение:

\[
\exp(x_1) + 2(x_1 + (-x_1)) = \exp(x_1) = 0.
\]

Так как \( \exp(x_1) > 0 \) для всех \( x_1 \), стационарных точек в конечной области нет. Подставим \( x_2 = -x_1 \) в функцию:

\[
f(x_1, -x_1) = \exp(x_1) + (x_1 + (-x_1))^2 = \exp(x_1).
\]

Производная \( g(x_1) = \exp(x_1) \):

\[
\frac{d}{dx_1} \exp(x_1) = \exp(x_1),
\]

которая никогда не равна нулю. Однако \( \exp(x_1) \) монотонно убывает при \( x_1 \to -\infty \), и \( f(x_1, -x_1) \to 0 \) при \( x_1 \to -\infty \), \( x_2 = -x_1 \to +\infty \).

\textbf{Вывод}: Функция не имеет конечного минимума. Минимальное значение \( f \to 0 \) достигается при \( x_1 \to -\infty \), \( x_2 = -x_1 \).

\section{Реализация}

\lstinputlisting[language=go, caption=Метод наискорейшего спуска]{code/lab5.go}

\section{Результаты}

\begin{table}[H]
    \centering
    \caption{Результаты метода наискорейшего спуска}
    \begin{tabular}{lccc}
    \toprule
    Метод & Точка минимума & Значение \( f(x) \) & Итераций \\
    \midrule
    Наискорейший спуск & \( (-6.4446, 6.4441) \) & 0.001589 & 586 \\
    Аналитическое & \( x_1 \to -\infty \), \( x_2 = -x_1 \) & \( \to 0 \) & — \\
    Пример (аналитическое) & \( (-10.0, 10.0) \) & 0.000045 & — \\
    \bottomrule
    \end{tabular}
    \label{tab:results}
\end{table}

\section{Вывод}

Метод наискорейшего спуска успешно нашёл приближение к минимуму функции с точностью \( \varepsilon = 0.001 \). 
Численное решение \( x_1 \approx -6.4446 \), \( x_2 \approx 6.4441 \), \( f \approx 0.001589 \) соответствует аналитическому поведению: 
\( x_1 \to -\infty \), \( x_2 = -x_1 \), \( f \to 0 \). Для сравнения выбрана точка \( x_1 = -10 \), \( x_2 = 10 \), где \( f \approx 0.000045 \). 
Разница между численным и аналитическим значением функции составила \( 0.001544 \), что близко к заданной точности \( \varepsilon \).

Метод наискорейшего спуска потребовал 586 итераций, что указывает на медленную сходимость. Использование метода парабол для поиска \( t^* \) обеспечило 
точное определение направления спуска. 
Графический анализ и аналитическое решение подтверждают отсутствие конечного минимума, но численный метод позволяет найти приближение, пригодное для 
практических целей.

\end{document}